\section{Exploração dos Dados}

\begin{frame}{Exploração dos Dados}
A base disponibilizada pela Prefeitura do Recife possui as seguintes
características:
\begin{itemize}
	\item Registro do fluxo de veículos nas vias mais movimentadas do Recife;
	\item A granularidade dos dados é de número de veículos em um intervalo
	de 15 minutos;
	\item O período de medições foi entre 8 de junho de 2015 e 12 de junho
	de 2015;
	\item A CTTU é o responsável pelos dados, que foram disponibilizados
	pela Emprel;
\end{itemize}
\end{frame}

\begin{frame}{Exploração dos Dados}
Os campos disponíveis são:
\begin{itemize}
	\item \textbf{Logradouro}: onde o equipamento de monitoração estava
	instalado;
	\item \textbf{Data}: a data de quando ocorreu cada monitoração;
	\item \textbf{Hora}: o intervalo no qual a medição foi feita;
	\item \textbf{Caminhões Grandes}: a quantidade de caminhões grandes que
	passaram;
	\item \textbf{Caminhões Médios/Grandes}
	\item \textbf{Caminhões Médios/Pequenos}
	\item \textbf{Caminhões Pequenos}
	\item \textbf{Motocicletas}
\end{itemize}
\end{frame}

\begin{frame}{Exploração dos Dados}
Os campos disponíveis são:
\begin{itemize}
	\item \textbf{Não Reconhecido}
	\item \textbf{Ônibus Grande}
	\item \textbf{Ônibus Pequeno}
	\item \textbf{Passeio Grande}
	\item \textbf{Passeio Pequeno}
\end{itemize}
\end{frame}

\begin{frame}{Exploração dos Dados}
As vias contempladas foram: Avenida Boa Viagem, Rua Arquiteto Luiz Nunes,
Avenida General San Martin, Avenida Afonso Olindense, Avenida Maurício de
Nassau, Avenida Alfredo Lisboa, Rua Capitão Temudo (Cabanga, em frente à Praça
Governador Paulo Guerra), Avenida Antônio Goes, Avenida Dom João VI, Avenida
Saturnino de Brito, Avenida Governador Agamenon Magalhães, Avenida Mascarenhas
de Moraes, Avenida Conselheiro Aguiar, Avenida Engenheiro José Estelita, Avenida
Cais Santa Rita, Avenida Rui Barbosa.
\end{frame}

\begin{frame}{Exploração dos Dados}
Os últimos 10 campos foram somados na computação do fluxo total da via no
período de medição.
\vskip 0.5cm
A base foi particionada em $20$ amostras, uma para cada logradouro, de tamanhos
diferentes (pois há valores faltando, ou \textit{missing values}).
\vskip 0.5cm
Nos slides seguintes, serão mostrados os histogramas, \textit{boxplots} e
\textit{Q-Q plots} de cada uma das amostras, assim como o p-valor para o teste
de normalidade Shapiro-Wilk.
\end{frame}


\begin{frame}{Exploração dos Dados}
Avenida Boa Viagem, entre os números 6114 e 5888:
\vskip 0.05cm
\begin{figure}
	\centering
	\begin{subfigure}{.33\textwidth}
		\centering
		\includegraphics[width=\linewidth]{../histograms/street1.png}
		\caption*{Histograma}
	\end{subfigure}%
	\begin{subfigure}{.33\textwidth}
		\centering
		\includegraphics[width=\linewidth]{../boxplots/street1.png}
		\caption*{\textit{Boxplot}}
	\end{subfigure}
	\begin{subfigure}{.32\textwidth}
		\centering
		\includegraphics[width=\linewidth]{../qqplots/street1.png}
		\caption*{\textit{QQ Plot}}
	\end{subfigure}
\end{figure}
\vskip 0.05cm
Média: $105.0699$; Desvio Padrão: $68.4856$; Mediana: $114.5$.

P-valor do Shapiro-Wilk: $1.1192 \times 10^{-17}$.
\end{frame}

\begin{frame}{Exploração dos Dados}
Avenida Boa Viagem, no terceiro jardim:
\vskip 0.05cm
\begin{figure}
	\centering
	\begin{subfigure}{.33\textwidth}
		\centering
		\includegraphics[width=\linewidth]{../histograms/street2.png}
		\caption*{Histograma}
	\end{subfigure}%
	\begin{subfigure}{.33\textwidth}
		\centering
		\includegraphics[width=\linewidth]{../boxplots/street2.png}
		\caption*{\textit{Boxplot}}
	\end{subfigure}
	\begin{subfigure}{.32\textwidth}
		\centering
		\includegraphics[width=\linewidth]{../qqplots/street2.png}
		\caption*{\textit{QQ Plot}}
	\end{subfigure}
\end{figure}
\vskip 0.05cm
Média: $149.22$; Desvio Padrão: $98.1323$; Mediana: $167$.

P-valor do Shapiro-Wilk: $1.7536 \times 10^{-17}$.
\end{frame}

\begin{frame}{Exploração dos Dados}
Rua Arquiteto Luiz Nunes, bairro da Imbiribeira, entre os números 314 e 375,
sentido IPSEP:
\begin{figure}
	\centering
	\begin{subfigure}{.33\textwidth}
		\centering
		\includegraphics[width=\linewidth]{../histograms/street3.png}
		\caption*{Histograma}
	\end{subfigure}%
	\begin{subfigure}{.33\textwidth}
		\centering
		\includegraphics[width=\linewidth]{../boxplots/street3.png}
		\caption*{\textit{Boxplot}}
	\end{subfigure}
	\begin{subfigure}{.32\textwidth}
		\centering
		\includegraphics[width=\linewidth]{../qqplots/street3.png}
		\caption*{\textit{QQ Plot}}
	\end{subfigure}
\end{figure}
Média: $193.7405$; Desvio Padrão: $122.8915$; Mediana: $197$.

P-valor do Shapiro-Wilk: $2.7082 \times 10^{-14}$.
\end{frame}

\begin{frame}{Exploração dos Dados}
Avenida General San Martin, número 1864:
\vskip 0.05cm
\begin{figure}
	\centering
	\begin{subfigure}{.33\textwidth}
		\centering
		\includegraphics[width=\linewidth]{../histograms/street4.png}
		\caption*{Histograma}
	\end{subfigure}%
	\begin{subfigure}{.33\textwidth}
		\centering
		\includegraphics[width=\linewidth]{../boxplots/street4.png}
		\caption*{\textit{Boxplot}}
	\end{subfigure}
	\begin{subfigure}{.32\textwidth}
		\centering
		\includegraphics[width=\linewidth]{../qqplots/street4.png}
		\caption*{\textit{QQ Plot}}
	\end{subfigure}
\end{figure}
\vskip 0.05cm
Média: $244.1841$; Desvio Padrão: $130.0874$; Mediana: $303$.

P-valor do Shapiro-Wilk: $2.224 \times 10^{-24}$.
\end{frame}

\begin{frame}{Exploração dos Dados}
Avenida Afonso Olindense, número 996:
\vskip 0.05cm
\begin{figure}
	\centering
	\begin{subfigure}{.33\textwidth}
		\centering
		\includegraphics[width=\linewidth]{../histograms/street5.png}
		\caption*{Histograma}
	\end{subfigure}%
	\begin{subfigure}{.33\textwidth}
		\centering
		\includegraphics[width=\linewidth]{../boxplots/street5.png}
		\caption*{\textit{Boxplot}}
	\end{subfigure}
	\begin{subfigure}{.32\textwidth}
		\centering
		\includegraphics[width=\linewidth]{../qqplots/street5.png}
		\caption*{\textit{QQ Plot}}
	\end{subfigure}
\end{figure}
\vskip 0.05cm
Média: $123.7571$; Desvio Padrão: $73.1166$; Mediana: $138$.

P-valor do Shapiro-Wilk: $2.2047 \times 10^{-14}$.
\end{frame}

\begin{frame}{Exploração dos Dados}
Avenida Maurício de Nassau, 276:
\vskip 0.05cm
\begin{figure}
	\centering
	\begin{subfigure}{.33\textwidth}
		\centering
		\includegraphics[width=\linewidth]{../histograms/street6.png}
		\caption*{Histograma}
	\end{subfigure}%
	\begin{subfigure}{.33\textwidth}
		\centering
		\includegraphics[width=\linewidth]{../boxplots/street6.png}
		\caption*{\textit{Boxplot}}
	\end{subfigure}
	\begin{subfigure}{.32\textwidth}
		\centering
		\includegraphics[width=\linewidth]{../qqplots/street6.png}
		\caption*{\textit{QQ Plot}}
	\end{subfigure}
\end{figure}
\vskip 0.05cm
Média: $193.3013$; Desvio Padrão: $123.7284$; Mediana: $206$.

P-valor do Shapiro-Wilk: $2.7647 \times 10^{-17}$.
\end{frame}

\begin{frame}{Exploração dos Dados}
Avenida Alfredo Lisboa, número 33:
\vskip 0.05cm
\begin{figure}
	\centering
	\begin{subfigure}{.33\textwidth}
		\centering
		\includegraphics[width=\linewidth]{../histograms/street7.png}
		\caption*{Histograma}
	\end{subfigure}%
	\begin{subfigure}{.33\textwidth}
		\centering
		\includegraphics[width=\linewidth]{../boxplots/street7.png}
		\caption*{\textit{Boxplot}}
	\end{subfigure}
	\begin{subfigure}{.32\textwidth}
		\centering
		\includegraphics[width=\linewidth]{../qqplots/street7.png}
		\caption*{\textit{QQ Plot}}
	\end{subfigure}
\end{figure}
\vskip 0.05cm
Média: $189.9832$; Desvio Padrão: $164.8623$; Mediana: $146$.

P-valor do Shapiro-Wilk: $1.6042 \times 10^{-19}$.
\end{frame}

\begin{frame}{Exploração dos Dados}
Rua Capitão Temudo, Cabanga, em frente à praça Governador Paulo Guerra, sentido
Pina:
\begin{figure}
	\centering
	\begin{subfigure}{.33\textwidth}
		\centering
		\includegraphics[width=\linewidth]{../histograms/street8.png}
		\caption*{Histograma}
	\end{subfigure}%
	\begin{subfigure}{.33\textwidth}
		\centering
		\includegraphics[width=\linewidth]{../boxplots/street8.png}
		\caption*{\textit{Boxplot}}
	\end{subfigure}
	\begin{subfigure}{.32\textwidth}
		\centering
		\includegraphics[width=\linewidth]{../qqplots/street8.png}
		\caption*{\textit{QQ Plot}}
	\end{subfigure}
\end{figure}
Média: $306.3838$; Desvio Padrão: $166.7222$; Mediana: $347$.

P-valor do Shapiro-Wilk: $2.0613 \times 10^{-16}$.
\end{frame}

\begin{frame}{Exploração dos Dados}
Avenida Antônio de Goes, após a Ponte Agamenon Magalhães, sentido Derby:
\begin{figure}
	\centering
	\begin{subfigure}{.33\textwidth}
		\centering
		\includegraphics[width=\linewidth]{../histograms/street9.png}
		\caption*{Histograma}
	\end{subfigure}%
	\begin{subfigure}{.33\textwidth}
		\centering
		\includegraphics[width=\linewidth]{../boxplots/street9.png}
		\caption*{\textit{Boxplot}}
	\end{subfigure}
	\begin{subfigure}{.32\textwidth}
		\centering
		\includegraphics[width=\linewidth]{../qqplots/street9.png}
		\caption*{\textit{QQ Plot}}
	\end{subfigure}
\end{figure}
Média: $401.8287$; Desvio Padrão: $223.2702$; Mediana: $474$.

P-valor do Shapiro-Wilk: $1.4037 \times 10^{-18}$.
\end{frame}

\begin{frame}{Exploração dos Dados}
Avenida Dom João VI, em frente ao Ponto de Ônibus, bairro da Imbiribeira:
\begin{figure}
	\centering
	\begin{subfigure}{.33\textwidth}
		\centering
		\includegraphics[width=\linewidth]{../histograms/street10.png}
		\caption*{Histograma}
	\end{subfigure}%
	\begin{subfigure}{.33\textwidth}
		\centering
		\includegraphics[width=\linewidth]{../boxplots/street10.png}
		\caption*{\textit{Boxplot}}
	\end{subfigure}
	\begin{subfigure}{.32\textwidth}
		\centering
		\includegraphics[width=\linewidth]{../qqplots/street10.png}
		\caption*{\textit{QQ Plot}}
	\end{subfigure}
\end{figure}
Média: $170.6666$; Desvio Padrão: $122.5965$; Mediana: $173$.

P-valor do Shapiro-Wilk: $1.9794 \times 10^{-16}$.
\end{frame}

\begin{frame}{Exploração dos Dados}
Avenida Saturnino de Brito, número 445, sentido Pina:
\vskip 0.05cm
\begin{figure}
	\centering
	\begin{subfigure}{.33\textwidth}
		\centering
		\includegraphics[width=\linewidth]{../histograms/street11.png}
		\caption*{Histograma}
	\end{subfigure}%
	\begin{subfigure}{.33\textwidth}
		\centering
		\includegraphics[width=\linewidth]{../boxplots/street11.png}
		\caption*{\textit{Boxplot}}
	\end{subfigure}
	\begin{subfigure}{.32\textwidth}
		\centering
		\includegraphics[width=\linewidth]{../qqplots/street11.png}
		\caption*{\textit{QQ Plot}}
	\end{subfigure}
\end{figure}
\vskip 0.05cm
Média: $193.7444$; Desvio Padrão: $135.966$; Mediana: $244$.

P-valor do Shapiro-Wilk: $1.123 \times 10^{-20}$.
\end{frame}

\begin{frame}{Exploração dos Dados}
Avenida Governador Agamenon Magalhães, próximo ao viaduto Presidente Médici:
\begin{figure}
	\centering
	\begin{subfigure}{.33\textwidth}
		\centering
		\includegraphics[width=\linewidth]{../histograms/street12.png}
		\caption*{Histograma}
	\end{subfigure}%
	\begin{subfigure}{.33\textwidth}
		\centering
		\includegraphics[width=\linewidth]{../boxplots/street12.png}
		\caption*{\textit{Boxplot}}
	\end{subfigure}
	\begin{subfigure}{.32\textwidth}
		\centering
		\includegraphics[width=\linewidth]{../qqplots/street12.png}
		\caption*{\textit{QQ Plot}}
	\end{subfigure}
\end{figure}
Média: $352.392$; Desvio Padrão: $219.4974$; Mediana: $359$.

P-valor do Shapiro-Wilk: $4.4718 \times 10^{-12}$.
\end{frame}

\begin{frame}{Exploração dos Dados}
Avenida Marechal Mascarenhas de Moraes, Aeroporto Bairro Imbiribeira:
\begin{figure}
	\centering
	\begin{subfigure}{.33\textwidth}
		\centering
		\includegraphics[width=\linewidth]{../histograms/street13.png}
		\caption*{Histograma}
	\end{subfigure}%
	\begin{subfigure}{.33\textwidth}
		\centering
		\includegraphics[width=\linewidth]{../boxplots/street13.png}
		\caption*{\textit{Boxplot}}
	\end{subfigure}
	\begin{subfigure}{.32\textwidth}
		\centering
		\includegraphics[width=\linewidth]{../qqplots/street13.png}
		\caption*{\textit{QQ Plot}}
	\end{subfigure}
\end{figure}
Média: $269.7269$; Desvio Padrão: $159.9477$; Mediana: $282$.

P-valor do Shapiro-Wilk: $3.5736 \times 10^{-17}$.
\end{frame}

\begin{frame}{Exploração dos Dados}
Avenida Conselheiro Aguiar, próximo ao número 1350, Conjunto Residencial
Pernambuco:
\begin{figure}
	\centering
	\begin{subfigure}{.33\textwidth}
		\centering
		\includegraphics[width=\linewidth]{../histograms/street14.png}
		\caption*{Histograma}
	\end{subfigure}%
	\begin{subfigure}{.33\textwidth}
		\centering
		\includegraphics[width=\linewidth]{../boxplots/street14.png}
		\caption*{\textit{Boxplot}}
	\end{subfigure}
	\begin{subfigure}{.32\textwidth}
		\centering
		\includegraphics[width=\linewidth]{../qqplots/street14.png}
		\caption*{\textit{QQ Plot}}
	\end{subfigure}
\end{figure}
Média: $318.4522$; Desvio Padrão: $218.1112$; Mediana: $324$.

P-valor do Shapiro-Wilk: $1.0549 \times 10^{-12}$.
\end{frame}

\begin{frame}{Exploração dos Dados}
Avenida Engenheiro Antônio de Goes, número 124:
\vskip 0.05cm
\begin{figure}
	\centering
	\begin{subfigure}{.33\textwidth}
		\centering
		\includegraphics[width=\linewidth]{../histograms/street15.png}
		\caption*{Histograma}
	\end{subfigure}%
	\begin{subfigure}{.33\textwidth}
		\centering
		\includegraphics[width=\linewidth]{../boxplots/street15.png}
		\caption*{\textit{Boxplot}}
	\end{subfigure}
	\begin{subfigure}{.32\textwidth}
		\centering
		\includegraphics[width=\linewidth]{../qqplots/street15.png}
		\caption*{\textit{QQ Plot}}
	\end{subfigure}
\end{figure}
\vskip 0.05cm
Média: $244.4404$; Desvio Padrão: $167.3475$; Mediana: $229.5$.

P-valor do Shapiro-Wilk: $8.5644 \times 10^{-20}$.
\end{frame}

\begin{frame}{Exploração dos Dados}
Avenida Engenheiro José Estelita, entre os postes de iluminação 19-R e 21-R:
\begin{figure}
	\centering
	\begin{subfigure}{.33\textwidth}
		\centering
		\includegraphics[width=\linewidth]{../histograms/street16.png}
		\caption*{Histograma}
	\end{subfigure}%
	\begin{subfigure}{.33\textwidth}
		\centering
		\includegraphics[width=\linewidth]{../boxplots/street16.png}
		\caption*{\textit{Boxplot}}
	\end{subfigure}
	\begin{subfigure}{.32\textwidth}
		\centering
		\includegraphics[width=\linewidth]{../qqplots/street16.png}
		\caption*{\textit{QQ Plot}}
	\end{subfigure}
\end{figure}
Média: $162.8684$; Desvio Padrão: $144.7380$; Mediana: $114$.

P-valor do Shapiro-Wilk: $2.7657 \times 10^{-21}$.
\end{frame}

\begin{frame}{Exploração dos Dados}
Avenida Cais Santa Rita, próximo ao número 675:
\vskip 0.05cm
\begin{figure}
	\centering
	\begin{subfigure}{.33\textwidth}
		\centering
		\includegraphics[width=\linewidth]{../histograms/street17.png}
		\caption*{Histograma}
	\end{subfigure}%
	\begin{subfigure}{.33\textwidth}
		\centering
		\includegraphics[width=\linewidth]{../boxplots/street17.png}
		\caption*{\textit{Boxplot}}
	\end{subfigure}
	\begin{subfigure}{.32\textwidth}
		\centering
		\includegraphics[width=\linewidth]{../qqplots/street17.png}
		\caption*{\textit{QQ Plot}}
	\end{subfigure}
\end{figure}
\vskip 0.05cm
Média: $425.8256$; Desvio Padrão: $311.6514$; Mediana: $327$.

P-valor do Shapiro-Wilk: $2.607 \times 10^{-23}$.
\end{frame}

\begin{frame}{Exploração dos Dados}
Avenida Rui Barbosa, número 1397:
\vskip 0.05cm
\begin{figure}
	\centering
	\begin{subfigure}{.33\textwidth}
		\centering
		\includegraphics[width=\linewidth]{../histograms/street18.png}
		\caption*{Histograma}
	\end{subfigure}%
	\begin{subfigure}{.33\textwidth}
		\centering
		\includegraphics[width=\linewidth]{../boxplots/street18.png}
		\caption*{\textit{Boxplot}}
	\end{subfigure}
	\begin{subfigure}{.32\textwidth}
		\centering
		\includegraphics[width=\linewidth]{../qqplots/street18.png}
		\caption*{\textit{QQ Plot}}
	\end{subfigure}
\end{figure}
\vskip 0.05cm
Média: $74.1692$; Desvio Padrão: $47.3340$; Mediana: $85.5$.

P-valor do Shapiro-Wilk: $2.7457 \times 10^{-16}$.
\end{frame}

\begin{frame}{Exploração dos Dados}
Rua Arquiteto Luiz Nunes, número 261A, sentido subúrbio:
\vskip 0.05cm
\begin{figure}
	\centering
	\begin{subfigure}{.33\textwidth}
		\centering
		\includegraphics[width=\linewidth]{../histograms/street19.png}
		\caption*{Histograma}
	\end{subfigure}%
	\begin{subfigure}{.33\textwidth}
		\centering
		\includegraphics[width=\linewidth]{../boxplots/street19.png}
		\caption*{\textit{Boxplot}}
	\end{subfigure}
	\begin{subfigure}{.32\textwidth}
		\centering
		\includegraphics[width=\linewidth]{../qqplots/street19.png}
		\caption*{\textit{QQ Plot}}
	\end{subfigure}
\end{figure}
\vskip 0.05cm
Média: $183.8888$; Desvio Padrão: $124.0896$; Mediana: $238$.

P-valor do Shapiro-Wilk: $2.6422 \times 10^{-11}$.
\end{frame}

\begin{frame}{Exploração dos Dados}
Rua Arquiteto Luiz Nunes, número 261A, sentido centro:
\vskip 0.05cm
\begin{figure}
	\centering
	\begin{subfigure}{.33\textwidth}
		\centering
		\includegraphics[width=\linewidth]{../histograms/street20.png}
		\caption*{Histograma}
	\end{subfigure}%
	\begin{subfigure}{.33\textwidth}
		\centering
		\includegraphics[width=\linewidth]{../boxplots/street20.png}
		\caption*{\textit{Boxplot}}
	\end{subfigure}
	\begin{subfigure}{.32\textwidth}
		\centering
		\includegraphics[width=\linewidth]{../qqplots/street20.png}
		\caption*{\textit{QQ Plot}}
	\end{subfigure}
\end{figure}
\vskip 0.05cm
Média: $181.7821$; Desvio Padrão: $146.9786$; Mediana: $141$.

P-valor do Shapiro-Wilk: $1.1984 \times 10^{-12}$.
\end{frame}


\begin{frame}{Exploração dos Dados}
	Algumas considerações sobre os dados:
	\vskip 0.5cm
	\begin{itemize}
		\item Desvio padrão alto para algumas amostras, por tanto, alta variabilidade.
		\item Nenhuma das amostras passou no teste de Shapiro-Wilk com nível de
		significância $\alpha = 0.05$, por tanto, não serão consideradas
		normais.
	\end{itemize}
\end{frame}