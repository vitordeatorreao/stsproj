\section{Criação do Ranking}

\begin{frame}{Definição Estatística}
Seja $X$ o número de carros que passam em frente a um logradouro durante um
intervalo de 15 minutos, então $X$ é uma variável aleatório \textbf{discreta}.

Foram extraídas $20$ amostras \textbf{aleatórias} e \textbf{independentes} de
$X$. Deseja-se ordenar as amostras a partir da média amostral levando em conta
a existência ou não de diferenças significativas entre elas. Assim, foram
utilizadas as seguintes hipóteses para cada par de amostras:

\vskip 0.5cm

$H_{0}$: Não existe diferença significativa entre as médias amostrais de duas
amostras de $X$.

$H_{1}$: Existe uma diferença significativa entre as médias amostrais de duas
amostras de $X$.
\end{frame}

\begin{frame}{Criação do Ranking}
Para criar o ranking, foram utilizadas as médias de cada uma das $20$ amostras.

\vskip 0.2cm

Para testar se a diferença entre as amostras é estatisticamente significante,
foi necessário selecionar um teste não-paramétrico, já que as amostras não são
normais.

\vskip 0.2cm

Assim, foi realizado um teste de Wilcoxon (não pareado) com significância
$\alpha = 0.05$. O teste foi aplicado para todos os pares de amostras.

\vskip 0.2cm

A função do R utilizada foi \texttt{wilcox.test} com parâmetro \texttt{paired}
igual a falso.
\end{frame}

\begin{frame}{Criação de Ranking}
Para construir o \textit{ranking}, primeiramente, as amostras foram ordenadas de
forma decrescente de acordo com seus respectivos fluxos médios de veículos.

Então, começando pela amostra com maior fluxo médio e seguindo na ordem, foram
realizados os seguintes passos:
\begin{enumerate}
	\item O \textit{rank} atual (começando pelo 1º) é atribuído a amostra;
	\item Todas as amostras que não possuem diferenças significativas (segundo o
	teste de Wilcoxon) para a amostra considerada e que ainda não foram
	adicionadas ao \textit{ranking}, são também adicionadas na mesma colocação;
	\item Se ainda existem amostras não inseridas no \textit{ranking}, passamos
	para a próxima amostra na ordem e para o próximo \textit{rank} e repetimos
	os passos.
\end{enumerate}
\end{frame}

\begin{frame}{Criação de Ranking}
O ranking final obtido foi:
\begin{enumerate}
	\item[1] Avenida Cais de Santa Rita, próximo ao número 675;
	\item[2] Avenida Antônio Goes, após a Ponte Agamenon Magalhães, sentido Derby;
	\item[3] Avenida Agamenon Magalhães, próximo ao viaduto Presidente Médici;
	\item[4] Avenida Conselheiro Aguiar, próximo número 1350;
	\item[4] Rua Capitão Temudo, Cabanga, sentido Pina;
	\item[5] Avenida Mascarenhas de Moraes, perto do Aeroporto;
	\item[6] Avenida Engenheiro Antônio Goes, número 124;
	\item[6] Avenida General San Martin, número 1864;
	\item[7] Avenida Saturtino de Brito, número 445;
\end{enumerate}
\end{frame}

\begin{frame}{Criação de Ranking}
O ranking final obtido foi:
\begin{enumerate}
	\item[7] Rua Arquiteto Luiz Nunes, entre número 314 e 375, sentido IPSEP;
	\item[7] Avenida Maurício de Nassau, número 276;
	\item[7] Avenida Alfredo Lisboa, número 33;
	\item[7] Rua Arquiteto Luiz Nunes, número 261A, sentido subúrbio;
	\item[7] Rua Arquiteto Luiz Nunes, número 261A, sentido centro;
	\item[8] Avenida Dom João VI;
	\item[8] Avenida Engenheiro José Estelita;
\end{enumerate}
\end{frame}

\begin{frame}{Criação de Ranking}
O ranking final obtido foi:
\begin{enumerate}
	\item[9] Avenida Boa Viagem, próximo ao terceiro jardim;
	\item[10] Avenida Afonso Olindense, número 996;
	\item[11] Avenida Boa Viagem, entre os números 6114 e 5888;
	\item[12] Avenida Rui Barbosa, número 1397;
\end{enumerate}
\end{frame}
