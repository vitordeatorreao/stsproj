\section{Conclusões}

\begin{frame}{Conclusões}
	\begin{itemize}
		\item A avenida Cais de Santa Rita é a que possui o maior fluxo médio de
		veículos.
		\item A avenida Rui Barbosa é a que possui menor fluxo médio de
		veículos.
		\item Não há diferenças significativas entre os fluxos médios de
		veículos entre as vias: Avenida Agamenon Magalhães e Rua Capitão Temudo,
		Avenida Saturnino de Brito e Rua Arquiteto Luiz Nunes, Arquiteto Luiz
		Nunes e Avenida Maurício de Nassau.
		\item Lembrando que ter um fluxo médio de veículos mais alto não
		significa, necessariamente, ser uma via mais engarrafada, e vice-versa.
	\end{itemize}
\end{frame}

\begin{frame}{Conclusões}
	\begin{itemize}
		\item Como vimos na comparação entre as amostras totais e em horários de
		pico, uma amostra grande não é necessariamente normal.
		\item As amostras do horário de pico são consideravelmente menores e, no
		entanto, muitas delas são normais (segundo o teste de Shapiro-Wilk),
		enquanto que as amostras maiores do primeiro experimento são todas não-
		normais.
	\end{itemize}
\end{frame}
