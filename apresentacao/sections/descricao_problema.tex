\section{Descrição do Problema}

\begin{frame}{Descrição do Problema}
	Todos os dias os cerca de 1,5 milhão de habitantes da zona metropolitana do Recife
    trafegam por entre as ruas e avenidas da cidade.
    \vskip 0.25cm
    Segundo o IBGE, a frota da cidade é uma das maiores do Brasil e é composta por mais
    de 600 mil veículos entre carros de passeio, ônibus, micro-ônibus, caminhonetes,
    caminhões e motocicletas.
    \vskip 0.25cm
    Em 2014, \href{http://www.em.com.br/app/noticia/nacional/2014/06/04/interna_nacional,536035/recife-lidera-indice-de-congestionamento-em-pesquisa-da-tomtom.shtml}{segundo a empresa TomTom},
    atuante no mercado de soluções para o tráfego, o Recife possuía o maior índice de 
    congestionamento do Brasil, com $60\%$ das suas vias congestionadas.
\end{frame}


\begin{frame}{Descrição do Problema}
	Diante desta situação, o que podem fazer os habitantes para se locomoverem evitando
    ao máximo o trânsito?
    \vskip 0.5cm
    \begin{itemize}
    \item Utilizar aplicativos de compartilhamentos de informações, como o Waze;
    \item Conhecer melhor o tráfego e o fluxo de veículos na sua cidade, para ter uma
    noção, sem a necessidade de consultas à bases de dados proprietárias, de quais vias
    devem ser evitadas durante seus percursos diários.
    \end{itemize}
\end{frame}
